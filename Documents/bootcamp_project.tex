\documentclass[11pt]{article}

\oddsidemargin=0.25truein \evensidemargin=0.25truein
\topmargin=-0.5truein \textwidth=6.0truein \textheight=8.75truein

%\RequirePackage{graphicx}
\usepackage{comment}
\usepackage{hyperref}
\urlstyle{rm}   % change fonts for url's (from Chad Jones)
\hypersetup{
    colorlinks=true,        % kills boxes
    allcolors=blue,
    pdfsubject={Data Bootcamp @ NYU Stern School of Business},
    pdfauthor={Dave Backus db3@nyu.edu},
    pdfstartview={FitH},
    pdfpagemode={UseNone},
%    pdfnewwindow=true,      % links in new window
%    linkcolor=blue,         % color of internal links
%    citecolor=blue,         % color of links to bibliography
%    filecolor=blue,         % color of file links
%    urlcolor=blue           % color of external links
% see:  http://www.tug.org/applications/hyperref/manual.html
}

%\renewcommand{\thefootnote}{\fnsymbol{footnote}}

% table layout
\usepackage{booktabs}

% section spacing and fonts
\usepackage[small,compact]{titlesec}

% list spacing
\usepackage{enumitem}
\setitemize{leftmargin=*, topsep=0pt}
\setenumerate{leftmargin=*, topsep=0pt, partopsep=0pt}

% attach files to the pdf
\usepackage{attachfile}
    \attachfilesetup{color=0.75 0 0.75}

\usepackage{needspace}
% example:  \needspace{4\baselineskip} makes sure we have four lines available before pagebreak

\usepackage{verbatim}

% change spacing of verbatim (not clear this does anything)
% http://tex.stackexchange.com/questions/43331/control-vertical-space-before-and-after-verbatim-environment
\usepackage{etoolbox}
\makeatletter
\preto{\@verbatim}{\topsep=0pt \partopsep=0pt}
\makeatother

% Make single quotes print properly in verbatim
\makeatletter
\let \@sverbatim \@verbatim
\def \@verbatim {\@sverbatim \verbatimplus}
{\catcode`'=13 \gdef \verbatimplus{\catcode`'=13 \chardef '=13 }}
\makeatother


% document starts here
\begin{document}
\parskip=\bigskipamount
\parindent=0.0in
\thispagestyle{empty}
{\large Data Bootcamp @ NYU Stern \hfill Coleman \& Lyon}


\bigskip\bigskip
\centerline{\Large \bf Data Bootcamp:  Project Guide}
\centerline{Revised: \today}

\section*{Overview}

One of our goals is for you to produce a piece of work you can use
to demonstrate your data and programming skills to potential employers.
That work will come in the form of a Jupyter notebook,
a format that combines code, text, and graphics in one user-friendly document.
We will add them to our GitHub repository so that
you can use a link to show others what you've done.
You will also be able to see what your classmates have done.

The relatively loose structure in self-directed projects like this
makes them more challenging than most things you do in school.
We think that also makes them more interesting.
They give you a chance to
indulge your curiosity and show off your creativity.

We have divided the project into components to keep you on track.
The intent is to make the project easier by breaking it down into a number
of small, manageable sub-projects.
The early steps are graded only on whether you do them:
you get 5 points if you do them, none if you don't.
We start with individual work.  Partway along, we encourage you to form groups
of two or three, but if you'd prefer to do this on your own,
that's ok, too.


\section*{Project outline}

The components of the project are
%
\begin{center}
\begin{tabular}{lllr}
\toprule
Assignment                  & Format  & Individual/Group &  Points \\
\midrule
Three Project Ideas         & Paper (hand-written)  & Individual  & 0  \\
Three Revised Project Ideas & Paper (professional)  & Individual  & 5  \\
Project Proposal            & Paper (professional)  & Group       & 5  \\
Data Report                 & Paper (professional)  & Group       & 5  \\
Final Project               & Jupyter notebook      & Group       & 85 \\
\bottomrule
\end{tabular}
\end{center}

The due dates are posted on the {course website}.
%{\bf Dates are firm and not open for negotiation.}


The components consist of:
\begin{itemize}

\item {\bf Three project ideas.}
Write down three project ideas in class.
One or two sentences each is enough.
Use your imagination.  Be creative.  Speak to others.
Write down things that interest you.
This will not be graded, but it will give you something to work with later on.

Over the coming weeks, we recommend you keep your ears open for possible
{\bf project ideas} --- and {\bf data sources\/}.


\item {\bf Three revised project ideas.}
After giving this more thought, write down three project ideas in (a little) more detail.
{\bf Include likely data sources.}
%Ask questions; we have a lot of experience with this, especially data.
All together this should be roughly one paragraph per idea and one page overall.
Like all work in this course, it should be clean and professional.
%We will keep the copy you hand in to use in class, so keep a separate copy for yourself.

{\bf We will not give you written feedback} on your ideas,
but {\bf we would be happy to talk in person}, either before class or another time.
Our experience is that exchanges like this work much better in person,
where we can bounce ideas around in ways that are hard to replicate in writing.

\item{\bf Project proposal.}
Form a group of between one and three --- no more ---
and choose a single project from those you submitted ---
or perhaps some other idea if you get a sudden flash of inspiration.
Flesh out the project in more detail, {\bf including the data source\/} and
two figures you plan to produce with it.
Total length should be no more than two pages.

On the off-chance you missed the point:  you should {\bf be clear about the data you plan to use\/}.
If you can't get the data, your whole project can be derailed.

{\bf Did we mention data?}

\item {\bf Data report.}
Describe your data and how you accessed it in enough detail that someone else could do it.
Yes, that's ``accessed,'' past tense.  You should have done this already to make sure you have it.
Include your Python code.  % if you read it directly from an internet source.
Internet sources are preferred, because it allows others to use your code and
follow up on your work.

\item {\bf Final project.}
You should submit your {\bf Jupyter notebook\/} to \href{mailto:nyu.databootcamp@gmail.com}{nyu.databootcamp@gmail.com}
by the due date listed on the course website.
The subject line should be:  ``bootcamp project ug'' or ``bootcamp project mba''
depending on the section you are in.
The file name should be your last names separated by dashes and a short title ---
something like {\tt Jones-Smith-Zhang-India.ipynb}.

Your project should include:
\begin{itemize}
\item Description cell.  Put a Markdown cell at the top of your notebook that
   includes the title of your project, a list of authors, and a short summary
  of what you do.  Think of the last one as an advertisement to potential readers.
\item Data.  Description of data sources and the code to read the data into Python
and reformat it as needed.
This should be done in enough detail that someone else can reproduce what you've done.
\item Graphics.  A series of figures that tell us something interesting.
Three or four would be enough, but do what you think works best for your project.
Think about the narrative:  What story do you want to tell?
\end{itemize}
%We'll post a template.
Here are a few \href{https://github.com/NYUDataBootcamp/Projects}
{past examples}.
\end{itemize}


\section*{Free advice}

Some things to keep in mind:
%
\begin{itemize}
\item {\bf Keep it simple.}
Most project ideas turn out to be too big.  You're generally well-advised
to carve out a manageable subset of what you think you can do.
There's no reason to worry about this at the idea generation stage,
but as you develop your project you may find that you need to focus
more narrowly on a part of it.

\item {\bf Find data.}  Make sure you can get the data you need.
One way to assure this is to start with data and ask what you can do with it.
Ideally you want the intersection (picture a Venn diagram) of an interesting
idea and good data.  You can start with either one, but ideas are often easier to
find than data.
Or start with an existing project that uses data you can access yourself
and extend it in some way.


\item {\bf Ask for help.}
We have years of experience with this kind of thing.
If you're stuck, let us know and we'll try to help.
You can also post questions on the class discussion board.

\end{itemize}


\section*{Grading}

Projects will be graded on their overall quality.  This includes, but is not restricted to,
these categories:
%
\begin{itemize}
\item Quality of the idea.  Is the question clearly articulated?  Is it interesting?
Does it have general appeal?
\item Quality of the data.  Does the data support the idea?
Is it the best data for this question?
\item Quality of the code. Is the code readable? Could someone else understand what you were doing and why?
\item Degree of difficulty.  Some ideas are harder than others to implement.
As in Olympic diving, you get credit for taking on a challenge.
\item Professional look.  Does your project look professional?  Are the graphs
easy to understand?  Are they clearly labeled?
\end{itemize}


{\vfill
{\bigskip \centerline{\it \copyright \ \number\year \
David Backus, Chase Coleman, and Spencer Lyon @ NYU Stern}%
}}



%\end{document}
\pagebreak
\thispagestyle{empty}
{\large Data Bootcamp @ NYU Stern \hfill Coleman \& Lyon}


\bigskip\bigskip
\centerline{\Large \bf Data Bootcamp:  Project Grade Sheet}
\centerline{Revised: \today}


\section*{Overall}

\begin{itemize} %\itemsep=0.75\bigskipamount
\item Clear idea and message?
\item Appropriate data?
\item Documentation of sources?
\item Effective graphics?
\item Professional appearance?
\end{itemize}
\bigskip

\section*{Idea}

\vspace*{0.75in}

%\begin{itemize} \itemsep=2\bigskipamount
%\item Clarity of the message
%\item ..
%\end{itemize}


\section*{Data}

\vspace*{0.75in}

%\begin{itemize} \itemsep=2\bigskipamount
%\item Clarity of the message
%\item ..
%\end{itemize}


\section*{Graphics}

\vspace*{0.75in}

%\begin{itemize} \itemsep=2\bigskipamount
%\item Clarity of the message
%\item ..
%\end{itemize}

\section*{Difficulty}

\vspace*{0.75in}

%\begin{itemize} \itemsep=2\bigskipamount
%\item Clarity of the message
%\item ..
%\end{itemize}

\section*{Professional look}

\vspace*{0.75in}

%\begin{itemize} \itemsep=2\bigskipamount
%\item Clarity of the message
%\item ..
%\end{itemize}


{\vfill
{\bigskip \centerline{\it \copyright \ \number\year \
David Backus, Chase Coleman, and Spencer Lyon @ NYU Stern}%
}}


\end{document}
