\documentclass[11pt]{exam}

\oddsidemargin=0.25truein \evensidemargin=0.25truein
\topmargin=-0.5truein \textwidth=6.0truein \textheight=8.75truein

%\RequirePackage{graphicx}
\usepackage{comment}
\usepackage{hyperref}
\urlstyle{rm}   % change fonts for url's (from Chad Jones)
\hypersetup{
    colorlinks=true,        % kills boxes
    allcolors=blue,
    pdfsubject={Data Bootcamp @ NYU Stern School of Business},
    pdfauthor={Dave Backus db3@nyu.edu},
    pdfstartview={FitH},
    pdfpagemode={UseNone},
%    pdfnewwindow=true,      % links in new window
%    linkcolor=blue,         % color of internal links
%    citecolor=blue,         % color of links to bibliography
%    filecolor=blue,         % color of file links
%    urlcolor=blue           % color of external links
% see:  http://www.tug.org/applications/hyperref/manual.html
}

%\renewcommand{\thefootnote}{\fnsymbol{footnote}}

% table layout
\usepackage{booktabs}

% section spacing and fonts
\usepackage[small,compact]{titlesec}

% list spacing
\usepackage{enumitem}
\setitemize{leftmargin=*, topsep=0pt}
\setenumerate{leftmargin=*, topsep=0pt, partopsep=0pt}

% attach files to the pdf
\usepackage{attachfile}
    \attachfilesetup{color=0.75 0 0.75}

\usepackage{needspace}
% example:  \needspace{4\baselineskip} makes sure we have four lines available before pagebreak

\usepackage{verbatim}

% change spacing of verbatim (not clear this does anything)
% http://tex.stackexchange.com/questions/43331/control-vertical-space-before-and-after-verbatim-environment
\usepackage{etoolbox}
\makeatletter
\preto{\@verbatim}{\topsep=0pt \partopsep=0pt}
\makeatother

% Make single quotes print properly in verbatim
\makeatletter
\let \@sverbatim \@verbatim
\def \@verbatim {\@sverbatim \verbatimplus}
{\catcode`'=13 \gdef \verbatimplus{\catcode`'=13 \chardef '=13 }}
\makeatother


%\printanswers

% document starts here
\begin{document}
\parskip=\bigskipamount
\parindent=0.0in
\thispagestyle{empty}
{\large Data Bootcamp @ NYU Stern \hfill Coleman \& Lyon}


\bigskip\bigskip
\centerline{\Large \bf Data Bootcamp:  Code Practice \#3}
\centerline{Revised: \today}

{\it Answer each of the questions below.
We recommend code with comments.
Please submit in hardcopy.}

The data input parts of each question are included in this template:

\vspace{-0.06in}
{\small
\url{https://raw.githubusercontent.com/NYUDataBootcamp/Materials/master/Code/Python/bootcamp_practice_3_template.py}
}
\vspace{-0.06in}

Or just navigate from the GitHub page:  \\
Code $\Rightarrow$ Python $\Rightarrow$ \verb|bootcamp_practice_3_template.py| $\Rightarrow$ click on Raw button


% \begin{solution}
% Short answers follows.
% The code is attached; download this pdf file,
% open it with the Adobe Reader or the equivalent,
% and click on the pushpin:
% \attachfile{../Code/Answers/bootcamp_practice_3_answerkey.py}
% \end{solution}

\begin{questions}
\item Enter and run this code in Spyder to produce the dataframe \texttt{weo}:
%\vspace{-0.15in}
\begin{verbatim}
import pandas as pd
data = {'BRA': [13.37, 13.30, 14.34, 15.07, 15.46, 15.98, 16.10],
        'JPN': [33.43, 31.83, 33.71, 34.29, 35.60, 36.79, 37.39],
        'USA': [48.30, 46.91, 48.31, 49.72, 51.41, 52.94, 54.60],
        'Year': [2008, 2009, 2010, 2011, 2012, 2013, 2014]}
weo  = pd.DataFrame(data)
\end{verbatim}
%\vspace{-0.15in}
The numbers are GDP per person in thousands of US dollars, 2008 to 2014,
variable PPPPC in the IMF's {\it World Economic Outlook\/} database.

\begin{parts}
\item Explain the \texttt{import} statement.
\item What type of object is {\tt data}?
\item Why does the last line have {\tt pd} prior to the {\tt DataFrame} function?
\item What type of object is {\tt weo}?
\item How many rows does it have?  Columns?
\item What {\tt dtypes} are the variables/columns?  What does this mean?
\item {\it Challenging.\/}
What are each of these expressions?  What type?
\begin{verbatim}
weo['Year']
weo[['Year']]
weo[[3]]
\end{verbatim}
\item {\it Challenging.\/}
Find and apply a method to convert {\tt weo['Year']} to type {\tt float}.
{\it Hint:\/} The method begins with the letter {\tt a}.
\item Describe the result of the statement \texttt{t = weo.tail(3)}.
What kind of object is \texttt{t}?  What does it look like?
\item How would you create a new dataframe that consists of the first 4 rows of \texttt{weo}?
\item What type of object is \texttt{weo['BRA']}?
\item Create a new variable equal to the ratio of Brazil's GDP per capita to Japan's.
\item {\it Challenging.\/}
Use the {\tt drop()} method to eliminate this (new) variable from the dataframe.
\item What are {\tt weo}'s row and column labels?
\item Set the index equal to the {\tt Year} variable.
\item Change the names of the other variables to Brazil, Japan, and United States.
\item Export the dataframe to an Excel spreadsheet.
\item What method would you use to compute the mean for each country?
What are the means?
\item {\it Challenging.\/}
How would you compute means across countries for each year?
\item Plot the data by applying a {\tt plot} method to {\tt weo}.
\item {\it Challenging.\/} Change the colors of the lines to green (Brazil), red (Japan),
and blue (US).
\item {\it Challenging.\/}
Do the same plot with a log scale.
{\it Hint:\/} Read the documentation for the plot method.
\item Plot Brazil on its own.
\end{parts}

\begin{solution}
\begin{parts}
\item The {\tt import} statement adds the {\tt pandas} data management package.
\item Dictionary --- note the curly braces.
\item The {\tt pd.} identifies {\tt DataFrame} as a pandas function.
\item Dataframe.
\item {\tt weo.shape} tells us that {\tt weo} has 7 rows and 4 columns.
\item {\tt weo.dtypes} generates the output
\begin{verbatim}
BRA     float64
JPN     float64
USA     float64
Year      int64
\end{verbatim}
So the first three variables are floats, the last one is an integer.
\item Apply the type function.  The first one is a series, the last two are
dataframes.
The difference here is the input.  If the input is a string, it returns a variable.
If the input is a list, it returns a dataframe.  The content is similar, but they're
(slightly) different objects.
This isn't particularly important, other than to remind ourselves that
the details matter here.

\item We use:  {\tt weo['Year'] = weo['Year'].astype(float)}.
The right side changes the type to float, then we assign it back to {\tt weo['Year']}.

\item {\tt t = weo.tail(3)}
is a dataframe consisting of the last three rows of {\tt weo}.

\item {\tt  weo.head(4)}.

\item A series, a single variable (column).

\item Use the statement:  {\tt weo['ratio'] = weo['BRA']/weo['JPN']}.

\item Use: {\tt weo = weo.drop(['ratio'], axis=1)}.

\item We get (and print) the row and column labels with
\begin{verbatim}
print('Row labels (index):', weo.index)
print('Column labels:', weo.columns)
\end{verbatim}

\item \verb|weo = weo.set_index(['Year'])|.

\item \verb| weo = weo.rename(columns={'BRA': 'Brazil', 'JPN': 'Japan', | \\
      \verb|                           'USA': 'United States'})|

\item \verb|weo.to_excel('weo.xls')|

\item {\tt print('Country means:', weo.mean(), sep='')}

\item {\tt print('Year means:', weo.mean(axis=1), sep='')}

\item {\tt weo.plot()}
(The year labels are messed up here, we'll have to fix that/)

\item {\tt weo.plot(color=['green', 'red', 'blue'])}

\item There's an ambiguity here over whether to use logs on the x axis, the y axis, or both.  This does the y axis:  \\ {\tt weo.plot(color=['green', 'red', 'blue'], logy=True)}

\item {\tt weo['Brazil'].plot()}

\end{parts}
\end{solution}

% ----------------------------------------------------------------------------
\item Use \verb|read_csv()| to read the responses of our class entry poll from

\vspace{0.1in}
\centerline{\url{http://pages.stern.nyu.edu/~dbackus/Data/Data-Bootcamp-entry-poll_s16.csv}}
\vspace{0.1in}

\begin{parts}
\item Read the file and assign it to the variable {\tt ep}.
\item Describe its contents.  What are the variables?  The responses?
\item What data types are the variables?
\item Change the variable names to something shorter.
\item {\it Challenging.\/}
Describe what this code does:
\begin{verbatim}
ep[list(ep)[1]].value_counts()
\end{verbatim}
{\it Suggestion:\/} Break it into two or more statements and explain them one at a time.
\end{parts}

\begin{solution}
\begin{parts}
\item \verb|ep = pd.read_csv(url)|
\item You can see them at the source.  The file is a ``dump''
of questions (the variable names) and answers (the data).
It's somewhat verbose, which makes it hard to print out and look at.

\item The dtypes are all objects:  strings.

\item The names are so long it's easier to replace all of them with
\begin{verbatim}
ep.columns = ['time', 'program', 'career', 'code_exp', 'stats_exp',
              'media', 'other', 'major', 'data', 'why', 'topics']
\end{verbatim}

\item This is practice with breaking complicated code into manageable pieces:
\begin{itemize}
\item {\tt list(ep)} is a list of variable names, each of them strings
\item {\tt list(ep)[1]} is the second variable name, \\
{\tt 'What program are you enrolled in?'},\\
or its replacement {\tt program}.
\item {\tt ep[list(ep)[1]]} is the second variable --- a series.
\item \verb|ep[list(ep)[1]].value_counts()|  applies the
\verb|value_counts()| method, which lists the number of each response to the question.  Here we get
\begin{verbatim}
MBA                             46
Undergraduate business          36
Other undergraduate             13
\end{verbatim}
and a collection of other responses with one or two each.

\end{itemize}

%\begin{verbatim}
%\end{verbatim}

\end{parts}
\end{solution}


% ----------------------------------------------------------------------------
\item  Consider the 538 college majors data at {\tt url}:
\begin{verbatim}
url1 = 'https://raw.githubusercontent.com/fivethirtyeight/data/master/'
url2 = 'college-majors/recent-grads.csv'
url = url1 + url2
\end{verbatim}
The variables are described at

\vspace{0.1in}
\centerline{\url{https://github.com/fivethirtyeight/data/tree/master/college-majors}}
\vspace{0.1in}

\begin{parts}

\item Create a dataframe {\tt df538} from the csv file at {\tt url} using
\verb|read_csv()|.
What are its dimensions?

\item What argument/parameter would you use to read only
the first ten lines of the file?

\item Extract the variables numbered \texttt{[2, 4, 15, 16, 17]}.
What are the names of these variables?
What do they represent?

\item Set the index equal to {\tt Major}.

\item Use the \verb|sort_values()| method to sort the data by {\tt Total}.

\item What code would you use to extract the ten majors with the greatest number of people?

\item {\it Challenging.\/}
Construct horizontal bar charts of the top ten majors sorted, first, by median salary
and, second, by the salary of the 25th percentile.
In each case plot just the variable you sorted on.
\end{parts}

\begin{solution}
\begin{parts}
\item Use the code:  \verb|df538 = pd.read_csv(url)|

\item {\tt nrows=10}.

\item The variables are total (the total number for this major),
median (median salary), P25th (25th percentile salary),
and P75th (75th percentile salary).
The last three describe the overall distribution of salaries:
the 25th, 50th, and 75th percentiles.

\item \verb|df = df.set_index(['Major'])|

\item \verb|h = df.head(10)|

\item \verb|h['Median'].sort_values().plot(kind='barh')|
\item \verb|h['P25th'].sort_values().plot(kind='barh')|

\end{parts}
\end{solution}

\item Approximately how long did this assignment take you?

\end{questions}



{\vfill
{\bigskip \centerline{\it \copyright \ \number\year \
David Backus, Chase Coleman, and Spencer Lyon @ NYU Stern}%
}}


\end{document}
